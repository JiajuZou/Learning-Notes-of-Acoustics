% sample root file for your "contribution" to a contributed volume
% Use this file as a template for your own input.
%
%%%%%%%%%%%%%%%% Springer %%%%%%%%%%%%%%%%%%%%%%%%%%%%%%%%%%


% RECOMMENDED %%%%%%%%%%%%%%%%%%%%%%%%%%%%%%%%%%%%%%%%%%%%%%%%%%%
\documentclass[graybox]{svmult}
\usepackage{type1cm}        % activate if the above 3 fonts are
                            % not available on your system
%
\usepackage{makeidx}         % allows index generation
\usepackage{graphicx}        % standard LaTeX graphics tool
                             % when including figure files
\usepackage{multicol}        % used for the two-column index
\usepackage[bottom]{footmisc}% places footnotes at page bottom

\usepackage{graphicx}


\usepackage{newtxtext}       % 
\usepackage{newtxmath}       % selects Times Roman as basic font

% see the list of further useful packages
% in the Reference Guide

\makeindex             % used for the subject index
                       % please use the style svind.ist with
                       % your makeindex program

%%%%%%%%%%%%%%%%%%%%%%%%%%%%%%%%%%%%%%%%%%%%%%%%%%%%%%%%%%%%%%%%%%%%%%%%%%%%%%%%%%%%%%%%%

\begin{document}

\title*{Learning Notes of Acoustics }
\author{Jiaju Zou}

\institute{Jiaju Zou \at Bachelor degree, SUSTech, Shenzhen China \\
                         Master student, BUAA, Beijing China \\
\email zoujj2020@mail.sustech.edu.cn}
\maketitle


\abstract{\\ 
\indent
This document is written mostly based on the learning notes of \textit{Sound and Sources of Sound}, a very suitable textbook for new learners to have a basic understanding of acoustics and aeroacoustics.\\
\indent
Classical acoustics theory will be discussed in this document, including the basic characteristics of sound, sound wave equations, sound propagation in pipes, sound wave at interfaces, Ray theory and sound generated near surfaces of discontinuity. Some aeroacoustic theories will also be discussed with a lower concentration, but we will never miss the important theory in this area, for example, the famous Lighthill Acoustic Analogy and the FW-H Equation.\\
\indent
Finally, some content in this document is simplified to focus on the important key points(these points are detailed and we promise you can understand them). Welcome to discuss with the author if you have any different understanding about the theories in this document since the author is still learning in this area.
}

\section{Characteristics of Sound}
\label{sec:1}
\subsection{Introduction}
\label{subsec:1}

\textbf{Sound}
\begin{enumerate}
    \item Elastic medium (gas, water, solid) is the necessary condition of sound propagation, therefore sound cannot propagate in the vacuum.
    \item Sound wave propagation speed $c_0$ : in air $\approx 340 \, \text{m/s}$ , in solid $\approx 1500 \, \text{m/s}$.
    \item Molecules in the medium (fluid particles) vibrate at the speed of around $0.1 \, \text{m/s}$.
    \item Longitudinal wave motion (the direction of particle vibration is the same as the direction of the wave propagation).
\end{enumerate}

\bigskip

\noindent 
\textbf{Mean State}\\ 
First, let's have a look at the following parameters:\\
 \textbullet\ ambient pressure $P_0$ \\
 \textbullet\ perturbation pressure $p'(\mathbf{x}, t)$\\
 \textbullet\ ambient density $\rho_0$ \\
 \textbullet\ perturbation density $\rho'(\mathbf{x}, t)$\\
Then the total unsteady pressure and density can be written as:
\begin{center}
$P_0 + p'(\mathbf{x}, t)$ \\
$\rho_0 + \rho'(\mathbf{x}, t)$
\end{center}

Now consider the perturbations can be negligible, the acoustic parameters satisfy the following condition:
\begin{center}
$p'(\mathbf{x}, t) \ll P_0$ \\
$\rho'(\mathbf{x}, t) \ll \rho_0$ \\
$v \ll c_0$\\
$s \ll \lambda $\\
\end{center}

where $v$ and $s$ are the speed and displacement of particle vibration separately , $\lambda$ is the wavelength of the propagating sound wave.

Mathematically, we can obtain the following results:
\begin{enumerate}
    \item The product of perturbation quantities can be negligible.
    \item The response of the acoustic field is linear.
    \item Sound can be regarded as linear motion.
\end{enumerate}

Here we list some common perturbation pressures in our daily life, which can help you have an intuitive understanding of the quantity of $p'(\mathbf{x}, t)$.\\
 \textbullet\ We human just feel the sound at 1000Hz: $2 \times 10^{-5} \, \text{Pa}$ \\
\textbullet\ The wind blowing the leaves: $2 \times 10^{-4} \, \text{Pa}$ \\
 \textbullet\ The talk in room with distance of 1m: $0.05 \sim 0.1 \, \text{Pa}$ \\

\noindent 
\textbf{Sound Pressure Level (SPL)}
\[
\text{SPL} = 20 \log_{10} \left( \frac{ \sqrt{\overline{p'^2}} }{2 \times 10^{-5} \text{Pa}} \right) \, \text{dB}
\]
Remarks:\\
\textbullet\ $p'$ is the pressure perturbation. \\
\textbullet\ $\sqrt{\overline{p'^2}}  $ is the root mean square of pressure perturbation, which can also be noted as $ p_{rms}$. \\
\textbullet\ As you can notice, the bottom part of the fraction is the value we human can just feel the sound at 1000Hz. \\
\textbullet\ If you do some acoustic measurement, the data you obtained throuth microphone is time domian signal, you have to do Fourier Transform (FT) to get the frequency domain result. (Actually, author is still trying to figure out this step, and there are some problems author can not explain very clearly... You can refer to section 1.2 to learn more about this step. The textbook \textit{Signals and Systems} is strongly suggested, which is the course the author missed as an undergraduate... )\\
\textbullet\ If the time domain signal $ p’ $ is simple harmonic, then you can also get $ p_{rms} $ through the following equation: 
\begin{center}
$ p_{rms} = p_e = \sqrt{\frac{1}{T} \int_{0}^{T}  p'^2\,dt } $
\end{center}
where $ p_e $ is the effective pressure.\\

\begin{legaltext}{Example 1.1}
\indent 
Consider two sound sources, amplitudes A and B, angular frequency $\omega_1 $ and $\omega_2 $, phase difference $ \sigma $ , SPL 85dB and 80dB. Let's have a look at the SPL at a given point in this sound field. \\

\noindent 
Source 1: $ p'_1 $ = $ Acos(\omega_1t) $ \\
Source 2: $ p'_2 $ = $ Bcos(\omega_2t + \sigma) $ \\
Total: $ p' $ = $ Acos(\omega_1t) + Bcos(\omega_2t + \sigma) $ \\
(We can add the two equation directly since the  linear sound wave assumption) \\

\noindent 
Square:
\begin{center}
$ p'^2 $=$ A^2cos^2(\omega_1t) + B^2cos^2(\omega_2t + \sigma) +AB[cos(\omega_1t+\omega_2t+\sigma) + cos(\omega_1t-\omega_2t-\sigma)] $ \\
\end{center}
Time Average: 
\begin{center}
$ \overline{p'^2} $ = $ \frac{1}{T} \int_{0}^{T}  p'^2\,dt $ = $ \frac{1}{2}A^2 + \frac{1}{2}B^2 + AB\overline{[\dots]}$
\end{center}

\begin{enumerate}
    \item If $ \omega_1 \neq \omega_2 $, $ AB\overline{[\dots]} = 0 $.
    \item If $ \omega_1 = \omega_2 $, $ AB\overline{[\dots]} = ABcos(\sigma) $
\end{enumerate}

You can also assume more conditions to make this problem more detailed, such as identical sound waves, only a single wave etc, and we promise you can find something interesting and useful.
\end{legaltext}

\label{sec:1}
\subsection{Sound Spectra}
\label{subsec:2}
\begin{center}
Spectral Analysis $ \Rightarrow $ Total Composition of Sound \\
Fourier transform (FT) is the key point.
\end{center} 

As mentioned before, if you do some acoustic measurement, the data you obtained through the microphone is a time-domain signal, you have to do Fourier Transform to get the frequency domain result. 

In this section, we will discuss this process in detail based on the author's understanding and knowledge. 

It's warmly welcomed to share your ideas and methods with us through the e-mail on the first page $ \text{:)} $

Now we list two equations as follows:
\begin{center}
$ F(\omega ) =  \int_{-\infty}^{+\infty} s(t)e^{i\omega t} \,dt  $ \\
$ s(t) = \frac{1}{2\pi} \int_{-\infty}^{+\infty} F(\omega)e^{i\omega t} \,d\omega  $
\end{center}

$  F(\omega ) $ is harmonic element, 

$ s(t) $ is signal function. \\

\begin{legaltext}{Example 1.2}
\indent 
Suppose one harmonic signal: $ s(t)=Ie^{i\omega_0 t} $ \\
\indent 
Its harmonic element: $ F(\omega )=I \int_{-\infty}^{+\infty} e^{i(\omega_0-\omega) t} \,dt = I\cdot2\pi\delta(\omega_0-\omega)$

\begin{figure}[h!]
    \centering
    \includegraphics[width=0.8\textwidth]
    {Fig.1.2.1.jpg} % 替换为你的图片文件名
    \caption{Harmonic signal and its element}
    \label{fig:example}
\end{figure}

In this example, we used the property of Delta-Function:
\begin{center}
$ \int_{-\infty}^{+\infty}g(t)\delta(t-t_0) \,dt = g(t_0) $ for $ \forall g(t) $
\end{center}

Do FT to the impulse signal $ I\delta(t-t_0)$ and use the above property, we can get:
\begin{center}
    $ F(\omega) = Ie^{-i\omega t_0} $
\end{center} 

Then, the following result can be derived:
\begin{center}
    $ s(t) = \frac{1}{2\pi} \int_{-\infty}^{+\infty} F(\omega)e^{i\omega t} \,d\omega  $ \\
    $ \delta(t-t_0) = \frac{1}{2\pi} \int_{-\infty}^{+\infty} e^{i\omega (t-t_0)} \,d\omega  $
\end{center}

\end{legaltext}

\section{Sound Wave Equations}
\label{sec:2}

\subsection{One-Dimensional Wave Equation}
\label{subsec:2}
In the section we will discuss the following situation of sound pressure perturbation:
\begin{center}
    $ p'=p'(x,t)$
\end{center}
Note that there is \textbf{ no bold} in the body of $ x $, which means $ x $ is only a scalar, and only one direction(for example, the plane wave). \\

We can obtain the following equation by combining the continuity equation of mass and momentum(ignore the viscous effect in sound waves):
\begin{center}
    $ \frac{\partial^2 \rho'}{\partial t^2} - \frac{\partial^2 p'}{\partial x^2} = 0 $
\end{center}

Then we have to consider the equations of thermodynamics. The speed of heat transfer is much lower than the speed of the volume change in the process of sound propagation, therefore we can regard this process as adiabatic, the pressure is only the function of density:
\begin{center}
    $ P=P(\rho)$
\end{center}

Expansion and linearization:
\begin{center}
    $ P=P_0 + (\rho - \rho_0)\frac{dP}{d\rho}(\rho_0) + ... $ \\
    $ \Downarrow $ \\ 
    $ P-P_0 = (\rho - \rho_0)\frac{dP}{d\rho}(\rho_0)$ \\
    $ \Downarrow $ \\ 
    $ p' = \rho'\frac{dP}{d\rho}(\rho_0)$
\end{center}

Introduce a new constant:
\begin{center}
    $ c^2 = \frac{dP}{d\rho}(\rho_0) = \frac{p'}{\rho'}$\\
    sound speed $ c $ \\
    $ p'$ travels with constant speed $ c $
\end{center}

Finally we can get:
\begin{center}
    $ \frac{1}{c^2} \frac{\partial^2 p'}{\partial t^2} - \frac{\partial^2 p'}{\partial x^2} = 0 $ \\
    1-D wave equation
\end{center}


General solution:
\begin{center}
    $ p'(x,t) = f(x-ct) +g(x+ct) $ \\
    $ f(x-ct) $ propagating to the right($ t\uparrow, x\uparrow$) \\
    $ g(x+ct) $ propagating to the left($ t\uparrow, x\downarrow $)
\end{center}

Harmonic Waves:
\begin{center}
    $ p'(x,t) = Ae^{i\omega (t-x/c)} + Be^{i\omega (t+x/c)}$ \\
    Only real part is meaningful.
\end{center}


The derivation of the 1-D wave equation is based on some assumptions(\textbf{inviscid, adiabatic, small perturbation}), and the result can be refered to as linear acoustics, which is still useful under lots of situations.

\subsubsection{Acoustic Particle velocity}
Instead of simply listing headings of different levels we recommend to let every heading be followed by at least a short passage of text.  Further on please use the \LaTeX\ automatism for all your cross-references and citations as has already been described in Sect.~\ref{subsec:2}, see also Fig.~\ref{fig:1}\footnote{If you copy text passages, figures, or tables from other works, you must obtain \textit{permission} from the copyright holder (usually the original publisher). Please enclose the signed permission with the manuscript. The sources\index{permission to print} must be acknowledged either in the captions, as footnotes or in a separate section of the book.}

Please note that the first line of text that follows a heading is not indented, whereas the first lines of all subsequent paragraphs are.

% For figures use
%
\begin{figure}[b]
\sidecaption
% Use the relevant command for your figure-insertion program
% to insert the figure file.
% For example, with the graphicx style use
\includegraphics[scale=.65]{figure}
%
% If no graphics program available, insert a blank space i.e. use
%\picplace{5cm}{2cm} % Give the correct figure height and width in cm
%
\caption{If the width of the figure is less than 7.8 cm use the \texttt{sidecapion} command to flush the caption on the left side of the page. If the figure is positioned at the top of the page, align the sidecaption with the top of the figure -- to achieve this you simply need to use the optional argument \texttt{[t]} with the \texttt{sidecaption} command}
\label{fig:1}       % Give a unique label
\end{figure}


\paragraph{Paragraph Heading} %
Instead of simply listing headings of different levels we recommend to let every heading be followed by at least a short passage of text.  Further on please use the \LaTeX\ automatism for all your cross-references and citations as has already been described in Sect.~\ref{sec:2}.

Please note that the first line of text that follows a heading is not indented, whereas the first lines of all subsequent paragraphs are.

For typesetting numbered lists we recommend to use the \verb|enumerate| environment -- it will automatically rendered in line with the preferred layout.

\begin{enumerate}
\item{Livelihood and survival mobility are oftentimes coutcomes of uneven socioeconomic development.}
\begin{enumerate}
\item{Livelihood and survival mobility are oftentimes coutcomes of uneven socioeconomic development.}
\item{Livelihood and survival mobility are oftentimes coutcomes of uneven socioeconomic development.}
\end{enumerate}
\item{Livelihood and survival mobility are oftentimes coutcomes of uneven socioeconomic development.}
\end{enumerate}


\subparagraph{Subparagraph Heading} In order to avoid simply listing headings of different levels we recommend to let every heading be followed by at least a short passage of text. Use the \LaTeX\ automatism for all your cross-references and citations as has already been described in Sect.~\ref{sec:2}, see also Fig.~\ref{fig:2}.

For unnumbered list we recommend to use the \verb|itemize| environment -- it will automatically be rendered in line with the preferred layout.

\begin{itemize}
\item{Livelihood and survival mobility are oftentimes coutcomes of uneven socioeconomic development, cf. Table~\ref{tab:1}.}
\begin{itemize}
\item{Livelihood and survival mobility are oftentimes coutcomes of uneven socioeconomic development.}
\item{Livelihood and survival mobility are oftentimes coutcomes of uneven socioeconomic development.}
\end{itemize}
\item{Livelihood and survival mobility are oftentimes coutcomes of uneven socioeconomic development.}
\end{itemize}

\begin{figure}[t]
\sidecaption[t]
% Use the relevant command for your figure-insertion program
% to insert the figure file.
% For example, with the option graphics use
\includegraphics[scale=.65]{figure}
%
% If no graphics program available, insert a blank space i.e. use
%\picplace{5cm}{2cm} % Give the correct figure height and width in cm
%
%\caption{Please write your figure caption here}
\caption{If the width of the figure is less than 7.8 cm use the \texttt{sidecapion} command to flush the caption on the left side of the page. If the figure is positioned at the top of the page, align the sidecaption with the top of the figure -- to achieve this you simply need to use the optional argument \texttt{[t]} with the \texttt{sidecaption} command}
\label{fig:2}       % Give a unique label
\end{figure}

\runinhead{Run-in Heading Boldface Version} Use the \LaTeX\ automatism for all your cross-references and citations as has already been described in Sect.~\ref{sec:2}.

\subruninhead{Run-in Heading Boldface and Italic Version} Use the \LaTeX\ automatism for all your cross-refer\-ences and citations as has already been described in Sect.~\ref{sec:2}\index{paragraph}.

\subsubruninhead{Run-in Heading Displayed Version} Use the \LaTeX\ automatism for all your cross-refer\-ences and citations as has already been described in Sect.~\ref{sec:2}\index{paragraph}.
% Use the \index{} command to code your index words
%
% For tables use
%
\begin{table}[!t]
\caption{Please write your table caption here}
\label{tab:1}       % Give a unique label
%
% Follow this input for your own table layout
%
\begin{tabular}{p{2cm}p{2.4cm}p{2cm}p{4.9cm}}
\hline\noalign{\smallskip}
Classes & Subclass & Length & Action Mechanism  \\
\noalign{\smallskip}\svhline\noalign{\smallskip}
Translation & mRNA$^a$  & 22 (19--25) & Translation repression, mRNA cleavage\\
Translation & mRNA cleavage & 21 & mRNA cleavage\\
Translation & mRNA  & 21--22 & mRNA cleavage\\
Translation & mRNA  & 24--26 & Histone and DNA Modification\\
\noalign{\smallskip}\hline\noalign{\smallskip}
\end{tabular}
$^a$ Table foot note (with superscript)
\end{table}
%
\section{Section Heading}
\label{sec:3}
% Always give a unique label
% and use \ref{<label>} for cross-references
% and \cite{<label>} for bibliographic references
% use \sectionmark{}
% to alter or adjust the section heading in the running head
Instead of simply listing headings of different levels we recommend to let every heading be followed by at least a short passage of text.  Further on please use the \LaTeX\ automatism for all your cross-references and citations as has already been described in Sect.~\ref{sec:2}.

Please note that the first line of text that follows a heading is not indented, whereas the first lines of all subsequent paragraphs are.

If you want to list definitions or the like we recommend to use the enhanced \verb|description| environment -- it will automatically rendered in line with the preferred layout.

\begin{description}[Type 1]
\item[Type 1]{That addresses central themes pertainng to migration, health, and disease. In Sect.~\ref{sec:1}, Wilson discusses the role of human migration in infectious disease distributions and patterns.}
\item[Type 2]{That addresses central themes pertainng to migration, health, and disease. In Sect.~\ref{subsec:2}, Wilson discusses the role of human migration in infectious disease distributions and patterns.}
\end{description}

\subsection{Subsection Heading} %
In order to avoid simply listing headings of different levels we recommend to let every heading be followed by at least a short passage of text. Use the \LaTeX\ automatism for all your cross-references and citations citations as has already been described in Sect.~\ref{sec:2}.

Please note that the first line of text that follows a heading is not indented, whereas the first lines of all subsequent paragraphs are.

\begin{svgraybox}
If you want to emphasize complete paragraphs of texts we recommend to use the newly defined class option \verb|graybox| and the newly defined environment \verb|svgraybox|. This will produce a 15 percent screened box 'behind' your text.

If you want to emphasize complete paragraphs of texts we recommend to use the newly defined class option and environment \verb|svgraybox|. This will produce a 15 percent screened box 'behind' your text.
\end{svgraybox}


\subsubsection{Subsubsection Heading}
Instead of simply listing headings of different levels we recommend to let every heading be followed by at least a short passage of text.  Further on please use the \LaTeX\ automatism for all your cross-references and citations as has already been described in Sect.~\ref{sec:2}.

Please note that the first line of text that follows a heading is not indented, whereas the first lines of all subsequent paragraphs are.

\begin{theorem}
Theorem text goes here.
\end{theorem}
%
% or
%
\begin{definition}
Definition text goes here.
\end{definition}

\begin{proof}
%\smartqed
Proof text goes here.
%\qed
\end{proof}

\paragraph{Paragraph Heading} %
Instead of simply listing headings of different levels we recommend to let every heading be followed by at least a short passage of text.  Further on please use the \LaTeX\ automatism for all your cross-references and citations as has already been described in Sect.~\ref{sec:2}.

Note that the first line of text that follows a heading is not indented, whereas the first lines of all subsequent paragraphs are.
%
% For built-in environments use
%
\begin{theorem}
Theorem text goes here.
\end{theorem}
%
\begin{definition}
Definition text goes here.
\end{definition}
%
\begin{proof}
%\smartqed
Proof text goes here.
%\qed
\end{proof}
%
\begin{trailer}{Trailer Head}
If you want to emphasize complete paragraphs of texts in an \verb|Trailer Head| we recommend to
use  \begin{verbatim}\begin{trailer}{Trailer Head}
...
\end{trailer}\end{verbatim}
\end{trailer}
%
\begin{question}{Questions}
If you want to emphasize complete paragraphs of texts in an \verb|Questions| we recommend to
use  \begin{verbatim}\begin{question}{Questions}
...
\end{question}\end{verbatim}
\end{question}
\eject%
\begin{important}{Important}
If you want to emphasize complete paragraphs of texts in an \verb|Important| we recommend to
use  \begin{verbatim}\begin{important}{Important}
...
\end{important}\end{verbatim}
\end{important}
%
\begin{warning}{Attention}
If you want to emphasize complete paragraphs of texts in an \verb|Attention| we recommend to
use  \begin{verbatim}\begin{warning}{Attention}
...
\end{warning}\end{verbatim}
\end{warning}

\begin{programcode}{Program Code}
If you want to emphasize complete paragraphs of texts in an \verb|Program Code| we recommend to
use

\verb|\begin{programcode}{Program Code}|

\verb|\begin{verbatim}...\end{verbatim}|

\verb|\end{programcode}|

\end{programcode}
%
\begin{tips}{Tips}
If you want to emphasize complete paragraphs of texts in an \verb|Tips| we recommend to
use  \begin{verbatim}\begin{tips}{Tips}
...
\end{tips}\end{verbatim}
\end{tips}
\eject
%
\begin{overview}{Overview}
If you want to emphasize complete paragraphs of texts in an \verb|Overview| we recommend to
use  \begin{verbatim}\begin{overview}{Overview}
...
\end{overview}\end{verbatim}

\end{overview}
\begin{backgroundinformation}{Background Information}
If you want to emphasize complete paragraphs of texts in an \verb|Background|
\verb|Information| we recommend to
use

\verb|\begin{backgroundinformation}{Background Information}|

\verb|...|

\verb|\end{backgroundinformation}|
\end{backgroundinformation}
\begin{legaltext}{Legal Text}
If you want to emphasize complete paragraphs of texts in an \verb|Legal Text| we recommend to
use  \begin{verbatim}\begin{legaltext}{Legal Text}
...
\end{legaltext}\end{verbatim}
\end{legaltext}
%
\begin{acknowledgement}
If you want to include acknowledgments of assistance and the like at the end of an individual chapter please use the \verb|acknowledgement| environment -- it will automatically be rendered in line with the preferred layout.
\end{acknowledgement}
%
\section*{Appendix}
\addcontentsline{toc}{section}{Appendix}
%
%
When placed at the end of a chapter or contribution (as opposed to at the end of the book), the numbering of tables, figures, and equations in the appendix section continues on from that in the main text. Hence please \textit{do not} use the \verb|appendix| command when writing an appendix at the end of your chapter or contribution. If there is only one the appendix is designated ``Appendix'', or ``Appendix 1'', or ``Appendix 2'', etc. if there is more than one.

\begin{equation}
a \times b = c
\end{equation}

\input{references}
\end{document}
